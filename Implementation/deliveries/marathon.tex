\documentclass[11pt,a4paper]{article}
\usepackage[utf8]{inputenc}
\usepackage[T1]{fontenc}
\usepackage[margin=1in]{geometry}
\usepackage{booktabs}
\usepackage{array}
\usepackage{hyperref}
\usepackage[brazil]{babel}

\title{Resultados da Maratona de Programação}
\date{\today}

\begin{document}
\maketitle

Neste relatório, discutimos os resultados e desafios acerca da Maratona de Programação proposta
para a disciplina de Metaheurísticas, ministrada pelo professor Igor Machado Coelho.

O objetivo foi atacar o problema de \textit{Dependência de Pacotes} utilizando metaheurísticas. Para
a solução apresentada, foram utilizados o \textbf{Biased Random Keys Genetic Algorithm (BRKGA)} e o \textbf{Variable
Neighborhood Descent (VND)}, contrapondo metaheurísticas populacionais àquelas baseadas em
vizinhanças de movimentos.

\section*{Resultados}

A seguir, apresentam-se as tabelas com os desempenhos das metaheurísticas. Para cada instância,
reportamos a \textbf{Pontuação} (função-objetivo) e o \textbf{Peso} (soma dos pesos das dependências utilizadas).

\subsection*{BRKGA}
\begin{table}[h]
  \centering
  \begin{tabular}{@{} l r r @{} }
    \toprule
    \textbf{Instância} & \textbf{Pontuação} & \textbf{Peso} \\
    \midrule
    sukp02\_100\_85\_0.10\_0.75.txt   & 11557 & 12010 \\
    sukp07\_285\_300\_0.10\_0.75.txt &  9494 & 38872 \\
    sukp28\_485\_500\_0.15\_0.85.txt  &  7590 & 71305 \\
    \bottomrule
  \end{tabular}
\end{table}

\subsection*{VND}
\begin{table}[h]
  \centering
  \begin{tabular}{@{} l r r @{} }
    \toprule
    \textbf{Instância} & \textbf{Pontuação} & \textbf{Peso} \\
    \midrule
    sukp02\_100\_85\_0.10\_0.75.txt   & 6797 & 11582 \\
    sukp07\_285\_300\_0.10\_0.75.txt & 3034 & 37608 \\
    sukp28\_485\_500\_0.15\_0.85.txt  & 2751 & 58502 \\
    \bottomrule
  \end{tabular}
\end{table}

\vspace{0.5em}
Todas as execuções foram limitadas a 90 segundos, conforme as regras da maratona.

As vizinhanças utilizadas no VND foram: \emph{Adicionar Pacote}, \emph{Retirar Pacote}, \emph{Adicionar Dependência} e
\emph{Retirar Dependência}.

Os resultados aqui apresentados passaram por correções pós-maratona (como acordado com o professor) e, por isso,
podem divergir dos números brutos da competição.

Observa-se que os resultados do BRKGA foram superiores aos do VND. Acreditamos que isso se deve, em grande parte,
à maturidade da implementação do algoritmo genético enviesado, à facilidade de calibração, ao fato de não depender
da engenharia de movimentos e ao bom desempenho inicial observado em comparação ao VND.

\end{document}
