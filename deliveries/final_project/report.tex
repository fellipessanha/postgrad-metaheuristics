\documentclass[12pt,a4paper]{article}

% Packages
\usepackage[utf8]{inputenc}
\usepackage[T1]{fontenc}
\usepackage[brazilian]{babel}
\usepackage[style=abnt]{biblatex}
\addbibresource{references.bib}
\usepackage{amsmath,amssymb,amsfonts}
\usepackage{graphicx}
\usepackage{booktabs}
\usepackage{hyperref}
\usepackage{algorithm}
\usepackage{algpseudocode}
\usepackage{listings}
\usepackage{xcolor}
\usepackage{geometry}
\usepackage{float}
\usepackage{multirow}

\geometry{margin=2.5cm}

% Code listing style
\lstset{
    basicstyle=\ttfamily\small,
    breaklines=true,
    frame=single,
    numbers=left,
    numberstyle=\tiny\color{gray},
    keywordstyle=\color{blue},
    commentstyle=\color{green!60!black},
    stringstyle=\color{orange},
}

% Title
\title{Metaheurísticas para Problemas de Otimização Combinatória:\\
Dependência de Pacotes e Mochila com Penalidades}
\author{Fellipe Sanha}
\date{\today}

\begin{document}

\maketitle

\begin{abstract}
    Este trabalho apresenta a implementação e análise de metaheurísticas para dois problemas de otimização
    combinatória: o Problema da Dependência de Pacotes e o Problema da Mochila com Penalidades (KPF). São exploradas
    abordagens de Programação Linear Inteira (ILP) e o algoritmo BRKGA (Biased Random-Key Genetic Algorithm), com
    comparação de desempenho em instâncias de diferentes tamanhos.
\end{abstract}

\tableofcontents
\newpage

\section{Introdução}

Neste trabalho abordaremos a descrição de variações de um NP difíceis, similar ao Problema da Mochila(Knapsack
Problem - KP), estratégias viáveis para resolver este problema em diferentes recortes temporais, e comparamos os
resultados obtidos frente aos resultados ótimos conhecidos na literatura.

\section{Problema Abordado}

O Problema da Mochila com Penalidades (Knapsack Problem with Forfeits - KPF) é uma variante do problema clássico da
mochila onde pares de itens podem ter penalidades associadas quando ambos são selecionados \cite{cerulli2020knapsack}.
\cite{moura2021ils} propuseram uma heurística ILS para este problema.

\subsection{Definição Formal}

Dado um conjunto de $n$ itens, onde cada item $i$ possui:
\begin{itemize}
    \item Peso $w_i$
    \item Lucro $p_i$
\end{itemize}

E um conjunto de pares de forfeit $F$, onde cada par $f = \{i, j\} \in F$ possui uma penalidade $d_f$ aplicada quando ambos os itens $i$ e $j$ são selecionados.

O objetivo é maximizar:
\begin{equation}
    \max \sum_{i=1}^{n} p_i x_i - \sum_{f \in F} d_f v_f
\end{equation}

Sujeito a:
\begin{align}
    \sum_{i=1}^{n} w_i x_i & \leq H                                   \\
    x_i + x_j - v_f        & \leq 1, \quad \forall f = \{i, j\} \in F \\
    x_i                    & \in \{0, 1\}, \quad \forall i            \\
    v_f                    & \in [0, 1], \quad \forall f \in F
\end{align}

\section{Soluções Exploradas}

\subsection{Programação Linear Inteira (ILP)}

A formulação ILP segue o modelo de Cerulli et al. \cite{cerulli2020knapsack}, onde:
\begin{itemize}
    \item $x_i = 1$ se o item $i$ é selecionado
    \item $v_f$ representa a ativação da penalidade do par de forfeit $f$
\end{itemize}

A implementação utiliza o solver HiGHS através da biblioteca JuMP em Julia, com warm-start baseado em soluções GRASP.

\subsection{BRKGA (Biased Random-Key Genetic Algorithm)}

O BRKGA (Biased Random-Key Genetic Algorithm) foi implementado com uma estratégia de codificação inteligente baseada em GRASP:

\begin{itemize}
    \item \textbf{Codificação}: Chaves aleatórias no intervalo $[0, 1]$
    \item \textbf{Decodificação}: Estratégia de threshold onde itens com chave $\geq$ threshold são candidatos à seleção
    \item \textbf{Warm-start}: Soluções iniciais geradas por heurística gulosa com diferentes valores de $\alpha$
\end{itemize}

\subsubsection{Parâmetros Utilizados}

\begin{table}[H]
    \centering
    \begin{tabular}{ll}
        \toprule
        \textbf{Parâmetro}   & \textbf{Valor} \\
        \midrule
        Tamanho da população & 1000           \\
        Threshold            & 0.5            \\

        $\alpha$ (GRASP)     & 0.7            \\
        Iterações máximas    & 50000          \\
        Tempo limite         & 60s            \\
        \bottomrule
    \end{tabular}
    \caption{Parâmetros do BRKGA}
\end{table}

\section{Comparação de resultados}

Nesta seção veremos os desempenhos das diferentes estratégias desenvolvidas. Comparamos, também, os resultados com os
ótimos conhecidos da literatura~\cite{moura2021ils}.

\subsection{Instâncias de Teste}

Foram escolhidas vinte instâncias no total, de quatro tamanhos diferentes, 500 itens, 700 itens, 800 itens, e 1000 itens
, visando avaliar o desempenho dos algoritmos desenvolvidos desde instâncias relativamente pequenas(500 itens) até
tamanhos considerados grandes(1000 itens).



\subsection{Comparação ILP vs BRKGA}

\begin{table}[H]
    \centering
    \small
    \begin{tabular}{cc|r|rr|rr|rr}
        \toprule
                              &                    &                & \multicolumn{2}{c|}{\textbf{60s}} & \multicolumn{2}{c|}{\textbf{120s}} & \multicolumn{2}{c}{\textbf{180s}}                                                   \\
        \textbf{Tamanho}      & \textbf{Instância} & \textbf{Ótimo} & ILP                               & BRKGA                              & ILP                               & BRKGA         & ILP             & BRKGA         \\
        \midrule
        \multirow{5}{*}{500}  & 1                  & 2626           & 2145                              & \textbf{2244}                      & \textbf{2336}                     & 2244          & \textbf{ 2336 } & 2244          \\
                              & 2                  & 2660           & \textbf{2308}                     & 2238                               & \textbf{2324}                     & 2238          & \textbf{2419}   & 2242          \\
                              & 3                  & 2516           & \textbf{2207}                     & 2152                               & \textbf{2299}                     & 2163          & \textbf{2310}   & 2198          \\
                              & 4                  & 2556           & \textbf{2233}                     & 2170                               & \textbf{2254}                     & 2170          & \textbf{2445}   & 2185          \\
                              & 5                  & 2625           & \textbf{2297}                     & 2199                               & \textbf{2353}                     & 2199          & \textbf{2409}   & 2218          \\
        \midrule
        \multirow{5}{*}{700}  & 1                  & 3589           & \textbf{3156}                     & 3059                               & \textbf{3218}                     & 3059          & 3\textbf{299}   & 3059          \\
                              & 2                  & 3679           & 2763                              & \textbf{2857}                      & 2775                              & 2\textbf{857} & \textbf{2913}   & 2857          \\
                              & 3                  & 3664           & \textbf{3128}                     & 3124                               & \textbf{3189}                     & 3124          & \textbf{3206}   & 3124          \\
                              & 4                  & 3647           & \textbf{3182}                     & 3085                               & \textbf{3267}                     & 3085          & \textbf{3267}   & 3085          \\
                              & 5                  & 3596           & \textbf{3188}                     & 3082                               & \textbf{3156}                     & 3082          & \textbf{3128}   & 3082          \\
        \midrule
        \multirow{5}{*}{800}  & 1                  & 4184           & \textbf{3400}                     & 3398                               & 3400                              & \textbf{3427} & \textbf{3520}   & 3427          \\
                              & 2                  & 4065           & 3300                              & \textbf{3371}                      & 3299                              & \textbf{3371} & 3299            & \textbf{3371} \\
                              & 3                  & 4104           & \textbf{3475}                     & 3316                               & \textbf{3475}                     & 3388          & \textbf{3475}   & 3388          \\
                              & 4                  & 4056           & 3218                              & \textbf{3349}                      & 3252                              & \textbf{3349} & 3320            & \textbf{3349} \\
                              & 5                  & 4086           & 3366                              & \textbf{3389}                      & 3366                              & \textbf{3389} & 3386            & \textbf{3389} \\
        \midrule
        \multirow{5}{*}{1000} & 1                  & 4940           & 3852                              & \textbf{4028}                      & 3916                              & \textbf{4077} & 3928            & \textbf{4077} \\
                              & 2                  & 4969           & 4034                              & \textbf{4105}                      & 4034                              & \textbf{4105} & 4034            & \textbf{4105} \\
                              & 3                  & 5177           & 4040                              & \textbf{4230}                      & 4211                              & \textbf{4230} & 4273            & \textbf{4230} \\
                              & 4                  & 5143           & 4091                              & \textbf{4235}                      & 4137                              & \textbf{4302} & 4212            & \textbf{4302} \\
                              & 5                  & 5136           & 4094                              & \textbf{4143}                      & 4019                              & \textbf{4145} & 4040            & \textbf{4145} \\
        \bottomrule
    \end{tabular}
    \caption{Resultados comparativos por tamanho de instância e tempo limite}
\end{table}

\begin{table}[H]
    \centering
    \small
    \begin{tabular}{c|rr|rr|rr}
        \toprule
                         & \multicolumn{2}{c|}{\textbf{60s}} & \multicolumn{2}{c|}{\textbf{120s}} & \multicolumn{2}{c}{\textbf{180s}}                                                    \\
        \textbf{Tamanho} & ILP                               & BRKGA                              & ILP                               & BRKGA          & ILP            & BRKGA          \\
        \midrule
        500              & \textbf{13.79}                    & 15.24                              & \textbf{10.88}                    & 15.15          & \textbf{8.17}  & 14.58          \\
        700              & \textbf{15.13}                    & 16.31                              & \textbf{14.1}                     & 16.31          & \textbf{12.97} & 16.31          \\
        800              & 18.23                             & \textbf{17.91}                     & 18.07                             & \textbf{17.42} & \textbf{17.06} & 17.42          \\
        1000             & 19.81                             & \textbf{18.23}                     & 20.22                             & \textbf{17.76} & 19.82          & \textbf{17.76} \\
        \bottomrule
    \end{tabular}
    \caption{GAP comparativo por tamanho de instância}
\end{table}

\section{Conclusão}

% TODO: Conclusões e trabalhos futuros

\printbibliography

\end{document}
